\documentclass{article}

\usepackage[english]{babel}
\usepackage[utf8]{inputenc}
\usepackage{amsmath,amssymb}
\usepackage{parskip}
\usepackage{graphicx}

% Margins
\usepackage[top=2.5cm, left=3cm, right=3cm, bottom=4.0cm]{geometry}

% custom footers and headers
\usepackage{fancyhdr}
\pagestyle{fancy}
\lhead{}
\chead{}
\rhead{}
\lfoot{Assignment 5}
\cfoot{}
\rfoot{Page \thepage}
\renewcommand{\headrulewidth}{0pt}
\renewcommand{\footrulewidth}{0pt}

\usepackage{multicol}
\usepackage{tikz}

% code listing settings
\usepackage{listings}
\usepackage{xcolor}

\definecolor{codegreen}{rgb}{0,0.6,0}
\definecolor{codegray}{rgb}{0.5,0.5,0.5}
\definecolor{codepurple}{rgb}{0.58,0,0.82}
\definecolor{backcolour}{rgb}{0.95,0.95,0.92}
\definecolor{framecolour}{rgb}{0.81,0.81,0.77}

\lstdefinestyle{mystyle}{
    aboveskip={1.0\baselineskip},
    belowskip={1.0\baselineskip},
    backgroundcolor=\color{backcolour},   
    commentstyle=\color{codegreen},
    keywordstyle=\color[rgb]{0.627,0.126,0.941},
    numberstyle=\tiny\color{codegray},
    stringstyle=\color{codepurple},
    basicstyle=\ttfamily\footnotesize,
    %numbers=left,
    frame=single,
    rulecolor=\color{framecolour},
    breakatwhitespace=false,         
    breaklines=true,                 
    captionpos=b,                    
    keepspaces=true,                 
    numbersep=5pt,                  
    showspaces=false,                
    showstringspaces=false,
    showtabs=false,                  
    tabsize=2
}

\lstset{style=mystyle}


%%%%%%%%%%%%%%%%%
%     Title     %
%%%%%%%%%%%%%%%%%
\title{DAT405 Assignment 5 -- Group XXX}
\author{Student 1 - (xxx hrs) \\[2pt]Student 2 - (yyy hrs)}
\date{\today}

\begin{document}
\maketitle

%%%%%%%%%%%%%%%%%
%   Problem 1   %
%%%%%%%%%%%%%%%%%
\section*{Problem 1}

Your text goes here! Remember to write in your own words and properly cite references, for instance like this \cite{myref1}. 

Properly emphasize and comment your code. You can add line numbers to the code by uncomment (removing the \% in front) the setting "numbers=left" in the header. 
Note also that you need to set code language in the listing environment to get the keywords properly color coded.


% code from http://rosettacode.org/wiki/Fibonacci_sequence#Python
\begin{lstlisting}[label={list:first},language=Python,caption=Sample Python code -- Fibonacci sequence.]
from math import *

# define function 
def analytic_fibonacci(n):
  sqrt_5 = sqrt(5);
  p = (1 + sqrt_5) / 2;
  q = 1/p;
  return int( (p**n + q**n) / sqrt_5 + 0.5 )
 
# define range
for i in range(1,31):
  print analytic_fibonacci(i)
\end{lstlisting}

%%%%%%%%%%%%%%%%%
%   Problem 2   %
%%%%%%%%%%%%%%%%%
\pagebreak
\section*{Problem 2}
Some more text and another reference \cite{myref2}.

The code below is in "language=bash" which is somewhat difficult to color code for LaTeX.

\begin{lstlisting}[label={list:second},language=bash,caption=Sample Bash code]
#! /bin/bash
python stage1.py
echo "Stage I done!"
python stage2.py
echo "Stage II done!"
python stage3.py
echo "Stage III done!"
\end{lstlisting}

\begin{thebibliography}{9}
\bibitem{myref1}
Donald E. Knuth (1986) \emph{The \TeX{} Book}, Addison-Wesley Professional.

\bibitem{myref2}
Leslie Lamport (1994) \emph{\LaTeX: a document preparation system}, Addison
Wesley, Massachusetts, 2nd ed.
\end{thebibliography}


\end{document}

